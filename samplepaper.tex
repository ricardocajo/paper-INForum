\documentclass[runningheads]{llncs}
%
\usepackage[T1]{fontenc}
% T1 fonts will be used to generate the final print and online PDFs,
% so please use T1 fonts in your manuscript whenever possible.
% Other font encondings may result in incorrect characters.
%
\usepackage{graphicx}
% Used for displaying a sample figure. If possible, figure files should
% be included in EPS format.
%
% If you use the hyperref package, please uncomment the following two lines
% to display URLs in blue roman font according to Springer's eBook style:
%\usepackage{color}
%\renewcommand\UrlFont{\color{blue}\rmfamily}
%
\begin{document}
%
\title{Formalization and Runtime Verification of Invariants for
Robotic Systems\thanks{Supported by organization x.}}
%
%\titlerunning{Abbreviated paper title}
% If the paper title is too long for the running head, you can set
% an abbreviated paper title here
%
\author{Ricardo Cordeiro\inst{1} \and
Alcides Fonseca\inst{1} \and
Christopher Timperley\inst{2}}
%\orcidID{0000-0002-8959-1702}
%
\authorrunning{Ricardo Cordeiro et al.}
% First names are abbreviated in the running head.
% If there are more than two authors, 'et al.' is used.
%
\institute{Faculdade de Ciências da Universidade de Lisboa,
Lisboa, Portugal \and
Carnegie Mellon University, Pittsburgh, PA}
%
\maketitle              % typeset the header of the contribution
%
\begin{abstract}
    Robotics has a big influence in today's society, so much that a potential failure in a robot may have extraordinary costs, not only financial, but can also cost lives.
    Current practices in robot testing are vast and involve such methods as simulations, log checking, or field tests. The frequent common denominator between these practices is the need for human visualization to determine the correctness of a given behavior. Automating this analysis could not only relieve this burden from a high-skilled engineer, but also allow for massive parallel executions of tests, that could potentially detect behavioral faults in the robots that would otherwise not be found due to human error or lack of time.
    We have developed a domain-specific language to specify properties of robotic systems in ROS. Specifications written by developers in this language can be compiled to a monitor ROS module, that will detect violations of those properties. We have used this language to express the temporal and positional properties of robots, and we have automated the monitoring of some behavioral violations of robots in relation to their state or events during a simulation.

\keywords{Robotics  \and Domain-specific language \and Runtime Monitoring \and Error detection \and Automation.}
\end{abstract}
%
%
%
\section{Introduction}

Robotics already have a great impact on our current society. Due to their broad practicality, the quality of software used by robots should be of extreme importance to us.

The Cyber-Physical systems of robots are non-deterministic and unreliable, mainly because robots interact directly with the real world. A sensor can return imprecise values since the environment itself can be very hard to predict. As a result, verifying whether a task or movement is correct can be hard for a system to conceive.

Current practices in testing robot software involve, field testing, simulation testing, logs checking, among others. The common denominator among these is that they require a human to analyze the behavior of the robot to determine whether the behavior is correct.

In the case of simulators, we can use the real value of objects' attributes in a simulation to compare with what the robot system perceives, but even so problems like what components to monitor and how to express them arise. Having a domain-specific language to specify a robotic system's properties can be useful or a burden depending on its complexity and accessibility.

This work has the objective of showing how a domain-specific language can be used to specify temporal and positional properties of robotic systems and monitor the simulation components associated with these properties.

The language allows describing a robotic system's properties in a somewhat simple and intuitive way, while at the same time still being able to express relevant temporal and positional arguments between robots and objects in the simulation. The language is supported by a compiler. The compiler translates the language to a monitoring mechanism. In this way, if a robotic system doesn't follow the properties defined by the user writing in the language, during execution, the compiler detects an anomaly and makes the analysis that the behavior of the robot is not consistent.

\section{Language}

The domain-specific language relies on an adaptation of linear temporal logic to express temporal relations of and between simulation objects.

The domain-specific language also has shortcuts to express the absolute values of certain useful concepts of objects in a simulation.

\subsection{Temporal Keywords}

\begin{itemize}
\item always X (X has to hold on the entire subsequent path);
\item never X (X never holds on the entire subsequent path);
\item eventually X (X eventually has to hold, somewhere on the subsequent path);
\item after X, Y (after the event X is observed, Y has to hold on the entire subsequent path);
\item until X, Y (X holds at the current or future position, and Y has to hold until that position. At that position, Y does not have to hold anymore);
\item after\_until X, Y, Z (after the event X is observed, Z has to hold on the entire subsequent path up until Y happens, at that position Z does not have to hold anymore);
\end{itemize}

\noindent It is also possible to reference previous variable states:
\begin{equation}
@\{X, -y\}
\end{equation}
This will represent the value of the variable X in the point in time -y.

\subsection{Useful Predicates}

\begin{itemize}
\item X.position (The position of the robot in the simulation);
\item X.position.y (The position in the y axis of the robot in the simulation. Also works for x and z);
\item X.distance.Y (The absolute distance between two objects in the simulation. For the x and y axis);
\item X.distanceZ.Y (The absolute distance between two objects in the simulation. For the x, y, and z axis);
\item X.velocity (The velocity of an object in the simulation. This refers to linear velocity);
\item X.velocity.x (The velocity in the x axis of an object in the simulation. This refers to linear velocity);
\item X.localization\_error - The difference between the robot's perception of its position and the actual position in the simulation;
\end{itemize}

\subsection{Examples}

As an example, we specify two properties of an arbitary autonomous driving robot:

\subsubsection{Property One}:

The robot velocity will never be above 2 in the duration of the simulation;

never robot.velocity > 2.0

\subsubsection{Property Two}:

The robot always needs to stop when coming near a stop sign;

after\_until robot.distance.stop\_sign < 1, robot.distance.stop\_sign > 1, eventually robot.velocity == 0

(Translating to a more human language we are saying that, after the robot distance to the stop\_sign is below the value of 1 in the simulator, up until the distance is again above 1, the robot velocity will eventually be equal to 0)

\section{Monitoring}
*Should i go in specifics about generated file? (subscribers, properties, etc..)?*
%%%%%%%%%%%%%%%%%%%%%%%%%%%%%%%%%%%%%%%%%%%%%%%%%%%%%%%%%%%%%%%%%%%%%%%%%%%%%%%%%%%
%\section{First Section}
%\subsection{A Subsection Sample}
%Please note that the first paragraph of a section or subsection is
%not indented. The first paragraph that follows a table, figure,
%equation etc. does not need an indent, either.

%Subsequent paragraphs, however, are indented.

%\subsubsection{Sample Heading (Third Level)} Only two levels of
%headings should be numbered. Lower level headings remain unnumbered;
%they are formatted as run-in headings.

%\paragraph{Sample Heading (Fourth Level)}
%The contribution should contain no more than four levels of
%headings. Table~\ref{tab1} gives a summary of all heading levels.

%\begin{table}
%\caption{Table captions should be placed above the
%tables.}\label{tab1}
%\begin{tabular}{|l|l|l|}
%\hline
%Heading level &  Example & Font size and style\\
%\hline
%Title (centered) &  {\Large\bfseries Lecture Notes} & 14 point, bold\\
%1st-level heading &  {\large\bfseries 1 Introduction} & 12 point, bold\\
%2nd-level heading & {\bfseries 2.1 Printing Area} & 10 point, bold\\
%3rd-level heading & {\bfseries Run-in Heading in Bold.} Text follows & 10 point, bold\\
%4th-level heading & {\itshape Lowest Level Heading.} Text follows & 10 point, italic\\
%\hline
%\end{tabular}
%\end{table}


%\noindent Displayed equations are centered and set on a separate
%line.
%\begin{equation}
%x + y = z
%\end{equation}
%Please try to avoid rasterized images for line-art diagrams and
%schemas. Whenever possible, use vector graphics instead (see
%Fig.~\ref{fig1}).

%\begin{figure}
%includegraphics[width=\textwidth]{fig1.eps}
%\caption{A figure caption is always placed below the illustration.
%Please note that short captions are centered, while long ones are
%justified by the macro package automatically.} \label{fig1}
%\end{figure}

%\begin{theorem}
%This is a sample theorem. The run-in heading is set in bold, while
%the following text appears in italics. Definitions, lemmas,
%propositions, and corollaries are styled the same way.
%\end{theorem}
%
% the environments 'definition', 'lemma', 'proposition', 'corollary',
% 'remark', and 'example' are defined in the LLNCS documentclass as well.
%
%\begin{proof}
%Proofs, examples, and remarks have the initial word in italics,
%while the following text appears in normal font.
%\end{proof}
%For citations of references, we prefer the use of square brackets
%and consecutive numbers. Citations using labels or the author/year
%convention are also acceptable. The following bibliography provides
%a sample reference list with entries for journal
%articles~\cite{ref_article1}, an LNCS chapter~\cite{ref_lncs1}, a
%book~\cite{ref_book1}, proceedings without editors~\cite{ref_proc1},
%and a homepage~\cite{ref_url1}. Multiple citations are grouped
%\cite{ref_article1,ref_lncs1,ref_book1},
%cite{ref_article1,ref_book1,ref_proc1,ref_url1}.

%\subsubsection{Acknowledgements} Please place your acknowledgments at
%the end of the paper, preceded by an unnumbered run-in heading (i.e.
%3rd-level heading).
%%%%%%%%%%%%%%%%%%%%%%%%%%%%%%%%%%%%%%%%%%%%%%%%%%%%%%%%%%%%%%%%%%%%%%%%%%%%%%%%%%%

%
% ---- Bibliography ----
%
% BibTeX users should specify bibliography style 'splncs04'.
% References will then be sorted and formatted in the correct style.
%
% \bibliographystyle{splncs04}
% \bibliography{mybibliography}
%
\begin{thebibliography}{8}
%\bibitem{ref_article1}
%Author, F.: Article title. Journal \textbf{2}(5), 99--110 (2016)

%\bibitem{ref_lncs1}
%Author, F., Author, S.: Title of a proceedings paper. In: Editor,
%F., Editor, S. (eds.) CONFERENCE 2016, LNCS, vol. 9999, pp. 1--13.
%Springer, Heidelberg (2016). \doi{10.10007/1234567890}

%\bibitem{ref_book1}
%Author, F., Author, S., Author, T.: Book title. 2nd edn. Publisher,
%Location (1999)

%\bibitem{ref_proc1}
%Author, A.-B.: Contribution title. In: 9th International Proceedings
%on Proceedings, pp. 1--2. Publisher, Location (2010)

%\bibitem{ref_url1}
%LNCS Homepage, \url{http://www.springer.com/lncs}. Last accessed 4
%Oct 2017
\end{thebibliography}
\end{document}
